\textbf{\textsc{Verduras}}

\begin{itemize}
 \item Calabacín | vitaminas (C; B3) y provitamina A %excelentes 
 para la salud y apariencia de la piel
 \item Cebolla | la alicina y la aliína tienen propiedades para reducir la tensión arterial, es antiinflamatorio, antioxidante, y sirve para favorecer la circulación.
 \item Choclo | tiene magnesio, 
 actúa como un relajante natural sobre el sistema nervioso, 
 vitaminas B1 B3 B9 actúan sobre el sistema nervioso, muscular e inmunológico. 
 %En casos de alergias e intolerancias se aconseja su consumo porque su aporte en 
 la fibra favorece la digestión y reduce el colesterol [recomendado para alergias e intolerancias]. 
%  El maíz posee 
 la vitamina A, antioxidante recomendado en la prevención de problemas inflamatorios
 \item Espinaca | carotenoides como betacarotenos, forma primaria de vitamina A, fuente de luteína con efectos antioxidantes. potasio hierro y folato
 \item Jengibre | activa digestión, mejora asimilación de nutrientes, con miel reduce náuseas, evita calambres, mejora flujo sanguíneo
 \item Menta | digestiva, como té ayuda cuando hay resfrío
 \item Pepino | contenido alto de agua, refrescante y poco energético [diurético poderoso]
 \item Remolacha | reduce presión arterial, óxido nítrico aporta impulso energético, potasio para corazón sano, manganeso para huesos hígado riñones y páncreas. [antiinflamatorio]
 \item Zanahoria | fuente de beta carotenos, la forma vegetal de la vitamina A
 \item Zapallito | 
 favorece al tratamiento de la gastritis, %ayuda a 
 baja colesterol alto y triglicéridos, 
 favorece la producción de glóbulos rojos y blancos, 
 ayuda a cuidar la salud de la vista, 
 mejora la salud de los huesos, dientes, piel y cabello

% Verduras
% Calabacín | vitaminas (C; B3 y provitamina A) excelentes para la buena salud y apariencia de la piel
% Cebolla | la alicina y la aliína tienen propiedades para reducir la tensión arterial, es antiinflamatorio, antioxidante, y sirve para favorecer la circulación.
% Choclo | tiene magnesio, actúa como un relajante natural sobre el sistema nervioso, vitaminas B1, B3 y B9, actúan sobre el sistema nervioso, muscular e inmunológico.
% En casos de alergias e intolerancias se aconseja su consumo porque su aporte en fibra favorece la digestión y reduce el colesterol.
% El maíz posee vitamina A, antioxidante recomendado en la prevención de problemas inflamatorios_
% Espinaca | carotenoides como betacarotenos, forma primaria de vitamina A, fuente de luteína con efectos antioxidantes. potasio hierro y folato
% Jengibre | activa digestión, mejora asimilación de nutrientes, con miel reduce náuseas, evita calambres, mejora flujo sanguíneo
% Menta | digestiva, como té ayuda cuando hay resfrío
% Pepino | contenido alto de agua, refrescante y poco energético [diurético poderoso]
% Remolacha | reduce presión arterial, óxido nítrico aporta impulso energético, potasio para corazón sano, manganeso para huesos hígado riñones y páncreas. [antiinflamatorio]
% Zanahoria | fuente de beta carotenos, la forma vegetal de la vitamina A
% Zapallito | favorece al tratamiento de la gastritis, ayuda a bajar el colesterol alto y los triglicéridos, favorece la producción de glóbulos rojos y blancos, ayuda a cuidar la salud de la vista, mejora la salud de los huesos, dientes, piel y cabello
% % Zapallo
% % Calabaza
% 
\end{itemize}